%----------------------------------------------------------------------------------------
%	PACKAGES AND OTHER DOCUMENT CONFIGURATIONS
%----------------------------------------------------------------------------------------

\documentclass[12pt]{article} % Default font size is 12pt, it can be changed here

\usepackage{geometry} % Required to change the page size to A4
\geometry{a4paper} % Set the page size to be A4 as opposed to the default US Letter

\usepackage{graphicx} % Required for including pictures

\usepackage{float} % Allows putting an [H] in \begin{figure} to specify the exact location of the figure
\usepackage{wrapfig} % Allows in-line images such as the example fish picture
\usepackage{psfrag,amsmath,amsfonts,amsthm,amssymb,cite}
\usepackage{lipsum} % Used for inserting dummy 'Lorem ipsum' text into the template

\linespread{1.2} % Line spacing

%\setlength\parindent{0pt} % Uncomment to remove all indentation from paragraphs

\graphicspath{{Pictures/}} % Specifies the directory where pictures are stored

\begin{document}

%----------------------------------------------------------------------------------------
%	TITLE PAGE
%----------------------------------------------------------------------------------------

\begin{titlepage}

\newcommand{\HRule}{\rule{\linewidth}{0.5mm}} % Defines a new command for the horizontal lines, change thickness here

\center % Center everything on the page

\textsc{\LARGE CS4516: Project Intro and Conclusion}\\[1.5cm] % Name of your university/college

\HRule \\[0.4cm]
{ \huge \bfseries Tag-Based IP Spoofing Prevention Under Low Deployment Scenarios}\\[0.4cm] % Title of your document
\HRule \\[1.5cm]

\begin{minipage}{0.4\textwidth}
\begin{flushleft} \large
\emph{Author:}\\
Michael Calder\\
Daniel Robertson\\
\end{flushleft}
\end{minipage}
~
\begin{minipage}{0.4\textwidth}
\begin{flushright} \large
\emph{Supervisor:} \\
Dr. Craig Shue % Supervisor's Name
\end{flushright}
\end{minipage}\\[4cm]

{\large \today}\\[3cm] % Date, change the \today to a set date if you want to be precise

%\includegraphics{Logo}\\[1cm] % Include a department/university logo - this will require the graphicx package

\vfill % Fill the rest of the page with whitespace

\end{titlepage}

%----------------------------------------------------------------------------------------
%	TABLE OF CONTENTS
%----------------------------------------------------------------------------------------

%\tableofcontents % Include a table of contents

\newpage % Begins the essay on a new page instead of on the same page as the table of contents 


%----------------------------------------------------------------------------------------
% -- Introduction --
%----------------------------------------------------------------------------------------

\section{Introduction}

A critical vulnerability exists within the very threading of the Internet. Our world's backbone for digital information exchange focuses naively on the receiver of a transmission, without giving proper regards to who the sender is. Malicious attackers exploit this short sight in order to temporarily disable target servers by issuing more requests then the target can handle, a technique known as denial of service (Dos). An attacker who rapidly alters, or spoofs, the source IP address on traffic they generate cannot be tracked. Existing counter measures attempt to mitigate the effect of DoS attacks on the server, but fail to address the underlying issue at hand. 

Shue et. al propose an eloquent solution which effectively curbs IP spoofing at the source. In their approach, implementing routers inspect the source address of each packet, and ensure that it is a conforming address with respect to their subnet\cite{Shue20081567}. Explicitly, if this router's subnet was 130.215.0.0/16 (that of the university this research was conducted under), and an encountered outgoing packet had a source IP of 176.230.1.5, the packet would be dropped. Should inspection pass, a unique tag (known by each other implementing router) which identifies the particular router, is added to the packet. Each downstream router en route to the destination checks for the presence of such a tag, and adds its own if one is found. Their analysis proves that their protocol is able to deny spoofed IP packets from entering the larger network at a rate nearing 100\%. The catch, however, is that this effectiveness is contingent on a high deployment status

Shue et. al note that the integrity of a given packet tag is weak under topologies with partial protocol deployment. A compromised tag threatens to undermine the efforts of the IP Spoofing prevention. Their own analysis shows that when only 10\% of the network implements their anti-spoofing protocol, then 66\% of networks can abuse a leaked tag (assuming 100\% collusion)\cite{Shue20081567}\footnote{Under more realistic levels of collusion (10\%), only 7\% of stolen tags can be reused effectively}. As the cost of changing router functionality to implement the protocol is likely to deter deployment speeds, the problem of tag theft may discourage adoption.

In an effort to smooth out the tag security concerns which emerge under low deployment scenarios, we propose two methods to secure the underlying packet tag implementation. Both combine a fast non-cryptographic hash algorithm (xxHash) with a nonce derived from the current unix time. By mending tag security concerns under low deployment, our modifications add further incentive to early adoption.




%----------------------------------------------------------------------------------------
%	Conclusion
%----------------------------------------------------------------------------------------

\bibliography{references.bib}
\bibliographystyle{plain}

%----------------------------------------------------------------------------------------

\end{document}
