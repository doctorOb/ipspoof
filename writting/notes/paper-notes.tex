\documentclass[12pt]{article}
	\usepackage{xcolor}
	\usepackage[tmargin=1in,bmargin=1in,lmargin=1.25in,rmargin=1.25in]{geometry}

\usepackage{setspace}
	\onehalfspacing

\begin{document}

	\bf{Needs}
	\begin{itemize}
		\item IP spoofing counter measures that play well with existing infrastructure
		\item Packet forwarding is done on destination IP, with no concern for source, which permits IP spoofing. 
			This faciliates numerous types of network attacks, such as Denial of Service and Man in the Middle. 
			A forwarding protocol which either eliminates or gravely reduces the likelihood of spoofing effectively 
			prevents these issues at the source.
		\item 
	\end{itemize}
	




	\section{Notes}
		\begin{itemize}
			\item Currently techniques aim to \italic{traceback} packet flows in order to identify the malicious source. Others try to protect the destination.
			\item Tags should not be changed often to avoid the overhead of relearning. But static tags present a possibility of theft. Perhaps, the tags could evolve over time, perhaps in every minute interval, with a set function being applied to them. This could mirror the use of unix intervals in the OTP schema used by google. Instead of {[}prefix, tag{]} being stored and indexed, it could be {[}prefix, raw_tag, mask{]}. Each tagging router would use bitwise operations on both the mask and the unix time interval to generate a tag which was only valid for the interval duration (on the order of 1-30s). This would ensure that if a tag was compromised, it would only be valid for a short duration. Only when an implementing router becomes nefarious, can the tag + mask be stolen.
		\end{itemize}



\end{document}