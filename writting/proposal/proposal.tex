%----------------------------------------------------------------------------------------
%	PACKAGES AND OTHER DOCUMENT CONFIGURATIONS
%----------------------------------------------------------------------------------------

\documentclass[12pt]{article} % Default font size is 12pt, it can be changed here

\usepackage{geometry} % Required to change the page size to A4
\geometry{a4paper} % Set the page size to be A4 as opposed to the default US Letter

\usepackage{graphicx} % Required for including pictures

\usepackage{float} % Allows putting an [H] in \begin{figure} to specify the exact location of the figure
\usepackage{wrapfig} % Allows in-line images such as the example fish picture
\usepackage{psfrag,amsmath,amsfonts,amsthm,amssymb,cite}
\usepackage{lipsum} % Used for inserting dummy 'Lorem ipsum' text into the template

\linespread{1.2} % Line spacing

%\setlength\parindent{0pt} % Uncomment to remove all indentation from paragraphs

\graphicspath{{Pictures/}} % Specifies the directory where pictures are stored

\begin{document}

%----------------------------------------------------------------------------------------
%	TITLE PAGE
%----------------------------------------------------------------------------------------

\begin{titlepage}

\newcommand{\HRule}{\rule{\linewidth}{0.5mm}} % Defines a new command for the horizontal lines, change thickness here

\center % Center everything on the page

\textsc{\LARGE CS4516: Project Proposal}\\[1.5cm] % Name of your university/college

\HRule \\[0.4cm]
{ \huge \bfseries Tag-Based IP Spoofing Prevention Under Low Deployment Scenarios}\\[0.4cm] % Title of your document
\HRule \\[1.5cm]

\begin{minipage}{0.4\textwidth}
\begin{flushleft} \large
\emph{Author:}\\
Daniel Robertson\\
Michael Calder\\
\end{flushleft}
\end{minipage}
~
\begin{minipage}{0.4\textwidth}
\begin{flushright} \large
\emph{Supervisor:} \\
Dr. Craig Shue % Supervisor's Name
\end{flushright}
\end{minipage}\\[4cm]

{\large \today}\\[3cm] % Date, change the \today to a set date if you want to be precise

%\includegraphics{Logo}\\[1cm] % Include a department/university logo - this will require the graphicx package

\vfill % Fill the rest of the page with whitespace

\end{titlepage}

%----------------------------------------------------------------------------------------
%	TABLE OF CONTENTS
%----------------------------------------------------------------------------------------

%\tableofcontents % Include a table of contents

\newpage % Begins the essay on a new page instead of on the same page as the table of contents 


%----------------------------------------------------------------------------------------
% -- Paper Outline --
% Executive Summary
% Needs
% Approach
% Benefits
% Competition
%----------------------------------------------------------------------------------------

\section{Executive Summary}
	The current approach to routing over the Internet Protocol enables an unfortunate opportunity for serious exploit. Packets are routed based off of destination prefix alone, and leave the source address field to only be considered by end hosts. Inherently, this means that an arbitrary source address could be used in place of a sender's true source address. Denial of Service attacks leverage this weakness by issuing dozens of requests using forged source addresses, so that a defending server can't know which requests can be ignored, and which are legitimate. Numerous solutions to mitigating DoS attacks exist, but they fail to guarantee absolute prevention. An absolute preventative solution would require the underlying Internet to conspire against IP spoofing, rather then source or end hosts. Shue et. al. present an elegant solution to the underlying problem, and show that it prevents a significant amount of spoofing attempts near the source, even under partial deployment\cite{Shue20081567}.

	In their approach, implementing routers within a prefix range ``mark'' packets originating from within their domains with a 48 bit tag. This acts as the router's signature and verification that the packet's source address is indeed within the router's prefix range. Upstream implementing routers verify these tags, strip them, and add their own.

\section{Need}
	Under full deployment, their analysis shows that this approach is effective in curbing IP spoofing attacks. However, there are notable pitfalls under partial and low deployment scenarios. As noted by Shue et al., both data and control plane modifications must be made to implementing routers. While no hardware modifications are required, the cost of changing router functionality, especially in core routers, is likely to deter deployment speeds. Tag theft can compromise the effectiveness of IP spoofing prevention. While their analysis shows that the protocol is highly resistant to spoofing and theft when fully deployed, partial deployment scenarios still suffer from a sizable security hole. Stronger security guarantees under low or long term deployment would further incentivize initial adoption. 

	The implementation detailed in \cite{Shue20081567} uses randomly generated fixed length tags. The verification process was designed to be fast, but introduces security concerns when tags are stolen. A safer approach might be to treat the tags as a secret, and use them to generate a safer token that can be sent over the wire and verified with the tag on the receiving end. Unfortunately, the more cryptographically secure the token is, the greater the incurred computational overhead becomes. For core routers, the added expense of token generation and verification is unacceptable, as forwarding decisions need to be a streamlined as possible. Any extra security added to the protocol would have to make compromises between cryptographic promises, and computational complexity.

\section{Approach}
	We present two possible avenues for research and analysis that could improve security for the packet tagging protocol presented by Shue et al., and help further incentivize early adoption. We aim to provide a stronger mechanism for protecting the underlying tags, such that the possibility and impact of tag theft are short reached and effectively negligible.

	Our first solution is to render stolen tags ineffective for further use. In this case, ``stolen'' refers to a tag that has not been leaked by a compromised router, but rather one that was intercepted during transit, or by an end host. We continue the assumption made by Shue et al. where any number of hosts may be malicious, but implementing routers are generally trustworthy. 

	We propose a mechanism inspired by the Time-based One Time Password (TOTP) algorithm, detailed in \cite{rfc6238}. TOTP uses fixed length intervals of unix time (usually 30 seconds) to generate a key that only remains valid for the duration of the interval. We aim to apply this time-based mechanism to the creation and verification of tokens, derived from a router's base tag. The interval used controls the duration of time that an intercepted tag-token could be exploited. Necessarily, it should be difficult to derive the base tag from an intercepted token. We will use a variety of cryptographic and non-cryptographic hashing algorithms in order to generate these tokens, and analyze their overhead and effectiveness, based on known metrics for evaluating hashes.

	Second, we plan to further secure the underlying tags using an evolving hash chain. Shue et al. suggest this as an appropriate means for changing a router's tag over time, but give no empirical or practical analysis of the effectiveness of such an alteration. We will follow the hash-chain generating approach given by Hu et al. \cite{Hu} in constructing the chains, and test its resilience when compromised. We also aim to qualify differing time intervals to advance the chain by. Finally, the chaining method will be combined with our previous altercations to enable an evolving collection of verifiable tags which are packaged for transport in a time-constrained and encrypted manner.

	Our implementation and analysis of this extended protocol will focus more heavily on tag exchange within a low deployment topology, where the assumption is made that tags have already been distributed and learned by deploying routers. As our proposal aims to add security mechanisms to an initialized network, we will not implement any of the tag discovery/exchange mechanisms detailed in \cite{Shue20081567}. Instead, our implementation will emulate an arbitrary exchange between two end hosts across a network whose deploying routers have fully built tag tables. The focus is thus on the overhead of differing token generating mechanism to the existing protocol, and the effectiveness of the added security. We will design a set of evaluation metrics to use for testing the strength of generated tokens, should they be stolen. This will likely include pre-image and collision resistance.

	We will use NS-3 to create and test our protocol. We plan to spend more time in its native C++ library, using the python bindings for high level simulations. 

\section{Benefits}

	The IP spoofing prevention methods provided by Shue et al. present a very strong case for adoption. Unfortunately, the security of their protocol performs poorly under low deployment, giving network operators less of an incentive to roll out their implementation. Our proposed extension aims to curb the possibility of tag theft and spoofing in partial deployment topologies. Added security assurances would further incentivize adoption. Increasing the out-of-the-box resilience of the protocol, especially while it is being rolled out, makes a better case for integration into current networks. 

\section{Competition} 
	
	Potential competition to our proposed extended protocol is the existing protocol its self. While our enhancements provide better security guarantees, especially under partial deployment (an area where these guarantees are lacking), if the cryptographic overhead proves to be too significant, then the original protocol may be favored. In addition, given the case where the entire Internet conspires to implement the original protocol, the issues coming form low deployment would be bypassed, and our extended protocol wouldn't be necessary.


%----------------------------------------------------------------------------------------
%	BIBLIOGRAPHY
%----------------------------------------------------------------------------------------

\bibliography{references.bib}
\bibliographystyle{plain}

%----------------------------------------------------------------------------------------

\end{document}
